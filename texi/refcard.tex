% -*- latex -*-
% Reference Card for Oort Gnus: to be processed with LaTeX2e
\documentclass{article}

\usepackage{supertabular}

\def\Guide{Card}\def\guide{card}
\def\logoscale{0.25}
\setlength{\textwidth}{7.26in} \setlength{\textheight}{10in}
\setlength{\topmargin}{-1.0in}
% the same settings work for A4, although there is a bit of space at the
% top and bottom of the page. 
\setlength{\oddsidemargin}{-0.5in} \setlength{\evensidemargin}{-0.5in}

\usepackage{epsfig}

% README:
% *** purpose
% this was originally thought of as a reference card (but as it is now 5+
% pages long, it may not be more useful than the online-help). It helps
% to get an overview for the Gnus-functionality.
%
% *** files
% refcard.tex (this file), gnusref.tex ("include"-file) and
% gnuslogo-refcard.eps (Gnus logo).
%
% *** printing (about 5 pages now: write me if you can make it shorter..)
% if you are using latex-mode, you do C-c C-f (process with latex),
% C-c C-v (view using xdvi/dvi2tty) and C-c C-p (print)
% the original author has set up the page dimensions cleverly so that this
% should print on both letter and a4 (see note above)
% $latex refcard.tex
% this creates a file refcard.dvi which you can preview using
% $xdvi refcard.dvi [C-c C-v]
% and print using something like
% $dvips refcard.dvi 
% which creates refcard.ps (print using 'lpr refcard.ps')
%
% *** customization:
% the part following \begin{document} in this file consists of a macro for
% each section and section-headers (\section*{..}). It should be easy to
% reorder things and/or remove sections (put '%' at the beginning of the line).
% (i.e. you might want to omit \notes in the printed version ?)
% If you think that the order is not logical and you have ideas for
% improvements, please send mail to the current maintainer.
%
% *** ChangeLog:
% 2000-03-26  Felix Natter  <fnatter@gmx.net>:
% refcard updated for Gnus 5.8.x: please send corrections or suggestions
% to the above email-address
% changes since 2000-03-26:
% o Create/Edit Foreign Groups: remove S b and S B (not available in 5.8.3)
% o Send/Reply etc.: remove w and W (the only bindings are S w and S W)
% Mon Apr  3 18:41:09 2000:
% o added C-c C-n and C-c C-t (Article)
% o C-c C-a as alias for M-m f (Article) + some other M-m *-bindings
% o added section for ``jumping'' in article-mode
% o now there's a difference between ``reading'' and ``composition''
% (article-modes)
% Apr 24th, 2000:
% o added D s, D S and D t for nndraft
% o group-mode: i.e. C-u RET does not actually fetch fewer articles; also ,=>;
% Fri Jul 14 23:15:43 2000:
% o added README-section
% Thu Jul 27 20:51:01 2000:
% o added Unplugged-commands
%
% *** TODO:
% o (LaTeX) how can you get 'tabular' to wrap around pages ?
% o some things might not be updated: scoring and server modes, maybe more
% o Gnus Unplugged category-buffer commands need to be written

\begin{document}
\newlength{\logowidth} \setlength{\logowidth}{6.861in}
\newlength{\logoheight} \setlength{\logoheight}{7.013in}

\def\progver{5.10}\def\refver{5.10-1} % program and refcard versions
\def\date{Oct 13th, 2001}
\def\author{Gnus Bugfixing Girls + Boys $<$bugs@gnus.org$>$}
\raggedbottom\raggedright

\twocolumn
% use \tiny to shrink it to 4 pages (needs a high-resaoultion printer, though)
%\tiny
\scriptsize
\pagestyle{plain}

% this contains a set of commands containing the actual sections
% (and some explanations).
% -*- mode: latex; TeX-master: "refcard.tex" -*-
%% include file for the Gnus refcard and booklet
%%
\newlength{\keycolwidth}
\newenvironment{keys}[1]% #1 is the widest key
{\nopagebreak%\noindent%
  \settowidth{\keycolwidth}{#1}%
  \addtolength{\keycolwidth}{\tabcolsep}%
  \addtolength{\keycolwidth}{-\columnwidth}%
  \begin{supertabular}{@{}l@{\hspace{\tabcolsep}}p{-\keycolwidth}@{}}}%
  {\end{supertabular}\\}

%% uncomment the first definition if you do not want pagebreaks in maps
%%\newcommand{\esamepage}{\samepage}
\newcommand{\esamepage}{}

\newcommand*{\B}[1]{{\bf#1})}    % bold l)etter

\newcommand{\Title}{%
  \begin{center}
    {\bf\LARGE Gnus \progver\ Reference \Guide\\}
                                %{\normalsize \Guide\ version \refver}
  \end{center}
  }

\newcommand*{\Logo}[1]{\centerline{%
    \makebox[\logoscale\logowidth][l]{\vbox to \logoscale\logoheight
      {\vfill\epsfig{figure=gnuslogo-#1}}\vspace{-\baselineskip}}}}

\newcommand{\Copyright}{%
  \begin{center}
    Copyright \copyright\ 1995 Free Software Foundation, Inc.\\*
    Copyright \copyright\ 1995 Vladimir Alexiev
    $<$vladimir@cs.ualberta.ca$>$.\\*
    Copyright \copyright\ 2000 Felix Natter $<$fnatter@gmx.net$>$.\\*
    Copyright \copyright\ 2001 \author.\\*
    Created from the Gnus manual Copyright \copyright\ 1994 Lars Magne
    Ingebrigtsen.\\*
    and the Emacs Help Bindings feature (C-h b).\\*
    Gnus logo copyright \copyright\ 1995 Luis Fernandes.\\*
  \end{center}

  Permission is granted to make and distribute copies of this reference
  \guide{} provided the copyright notice and this permission are preserved on
  all copies.  Please send corrections, additions and suggestions to the
  current maintainer's email address. \Guide{} last edited on \date.
  }

\newcommand{\Notes}{%
  \subsection*{Notes}
  {\esamepage
    Gnus is complex. Currently it has some 346 interactive (user-callable)
    functions. Of these 279 are in the two major modes (Group and
    Summary/Article). Many of these functions have more than one binding, some
    have 3 or even 4 bindings. The total number of keybindings is 389. So in
    order to save 40\% space, every function is listed only once on this
    \guide, under the ``more logical'' binding. Alternative bindings are given
    in parentheses in the beginning of the description.

    Many Gnus commands are affected by the numeric prefix. Normally you enter a
    prefix by holding the Meta key and typing a number, but in most Gnus modes
    you don't need to use Meta since the digits are not self-inserting. The
    prefixed behavior of commands is given in [brackets]. Often the prefix is
    used to specify:

    \quad [distance] How many objects to move the point over.

    \quad [scope] How many objects to operate on (including the current one).

    \quad [p/p] The ``Process/Prefix Convention'': If a prefix is given then it
    determines how many objects to operate on. Else if there are some objects
    marked with the process mark \#, these are operated on. Else only the
    current object is affected.

    \quad [level] A group subscribedness level. Only groups with a lower or
    equal level will be affected by the operation. If no prefix is given,
    `gnus-group-default-list-level' is used.  If
    `gnus-group-use-permanent-levels', then a prefix to the `g' and `l'
    commands will also set the default level.

    \quad [score] An article score. If no prefix is given,
    `gnus-summary-default-score' is used. \\*[\baselineskip]
                                % some keys
    Gnus startup-commands:
    \begin{keys}{M-x gnus-unplugged}
      M-x gnus           & start Gnus. \\
      M-x gnus-no-server & start Gnus without connecting to server
      (i.e. to read mail). \\
    \end{keys}
    Additionally, there are the two commands \texttt{gnus-plugged} and
    \texttt{gnus-unplugged}, which are only used if you want to download
    news and/or read previously downloaded news offline (see C-c C-i g Gnus
    Unplugged RET). Note: \texttt{gnus-no-server} ignores the stuff in
    \texttt{gnus-agent-directory}, and thus does not allow you to use Gnus
    Unplugged.
                                %
    \begin{keys}{C-c C-i}
      C-c C-i & Go to the Gnus online {\bf info}.\\
      C-c C-b & Send a Gnus {\bf bug} report.\\
    \end{keys}
    }}

\newcommand{\GroupLevels}{%
  The table below assumes that you use the default Gnus levels.
  Fill your user-specific levels in the blank cells.\\[1\baselineskip]
  \begin{tabular}{|c|l|l|}
    \hline
    Level & Groups & Status \\
    \hline
    1 & draft/mail groups   &              \\
    2 & mail groups         &              \\
    3 &                     & subscribed   \\
    4 &                     &              \\
    5 & default list level  &              \\
    \hline
    6 &                     & unsubscribed \\
    7 &                     &              \\
    \hline
    8 &                     & zombies      \\
    \hline
    9 &                     & killed       \\
    \hline
  \end{tabular}}

\newcommand{\MarkCharacters}{%
  {\esamepage If a command directly sets a mark, it is shown in parentheses.\\*
    \newlength{\markcolwidth}
    \settowidth{\markcolwidth}{` '}% widest character
    \addtolength{\markcolwidth}{4\tabcolsep}
    \addtolength{\markcolwidth}{-\columnwidth}
    \newlength{\markdblcolwidth}
    \setlength{\markdblcolwidth}{\columnwidth}
    \addtolength{\markdblcolwidth}{-2\tabcolsep}
    \begin{tabular}{|c|p{-\markcolwidth}|}
      \hline
      \multicolumn{2}{|p{\markdblcolwidth}|}{{\bf ``Read'' Marks.}
        All these marks appear in the first column of the summary line, and so
        are mutually exclusive.}\\ 
      \hline
      ` ' & (M-u, M SPC, M c) Not read.\\
      !   & (!, M !, M t) Ticked (interesting).\\
      ?   & (?, M ?) Dormant (only followups are interesting).\\
      E   & (E, M e, M x) {\bf Expirable}. Only has effect in mail groups.\\
      G   & (C, B DEL) Canceled article (or deleted in mailgroups).\\
      \hline\hline
      \multicolumn{2}{|p{\markdblcolwidth}|}
      {The marks below mean that the article
        is read (killed, uninteresting), and have more or less the same effect.
        Some commands however explicitly differentiate between them (e.g.\ M
        M-C-r, adaptive scoring).}\\
      \hline
      r   & (d, M d, M r) Deleted (marked as {\bf read}).\\
      C   & (M C; M C-c; M H; c, Z c; Z n; Z C) Killed by {\bf catch-up}.\\
      F   & SOUPed article. See the manual.\\
      O   & {\bf Old} (read in a previous session).\\
      K   & (k, M k; C-k, M K) {\bf Killed}.\\
      M   & Article marked as read by duplicate suppression.\\
      Q   & Article is part of a sparse thread (see ``Threading''
      in the manual).\\
      R   & {\bf Read} (viewed in actuality).\\
      X   & Killed by a kill file.\\
      Y   & Killed due to low score.\\
      \hline\multicolumn{2}{c}{\vspace{1ex}}\\\hline
      \multicolumn{2}{|p{\markdblcolwidth}|}
      {{\bf Marks not affecting visibility}}\\
      \hline
      \#  & (\#, M \#, M P p) Processable (will be affected by the next operation).
      [2]\\
      A   & {\bf Answered} (followed-up or replied). [2]\\
      *   & Cached. [2]\\
      S   & Saved. [2]\\
      +   & Over default score. [3]\\
      $-$ & Under default score. [3]\\
      $=$ & Has children (thread underneath it). Add `\%e' to
      `gnus-summary-line-format'. [3]\\
      \hline
    \end{tabular}
    }}

\newcommand{\GroupModeGeneral}{%
  \begin{keys}{C-c M-C-x}
    RET     & (=) Enter this group. [Prefix: how many (read) articles to fetch.
    Positive: newest articles, negative: oldest ones; non-numerical:
    view all articles, not just unread]\\
    M-RET   & Enter group quickly.\\
    M-SPC   & Same as RET but does not expunge and hide dormants.\\
    M-C-RET & Enter group without any processing, changes will not be permanent.\\
    SPC     & Select this group and display the first (unread) article. [Same
    prefix as above.]\\
    ?       & Give a very short help message.\\
    $<$     & Go to the beginning of the Group buffer.\\
    $>$     & Go to the end of the Group buffer.\\
    ,       & Jump to the lowest-level group with unread articles.\\
    .       & Jump to the first group with unread articles.\\
    xx & Enter the Server buffer mode.\\
    a       & Post an {\bf article} to a group.\\
    b       & Find {\bf bogus} groups and delete them.\\
    c       & Mark all unticked articles in this group as read ({\bf catch-up}).
    [p/p]\\ 
    g       & Check the server for new articles ({\bf get}). [level]\\
    M-g     & Check the server for new articles in this group ({\bf get}). [p/p]\\
    j       & {\bf Jump} to a group.\\
    m       & {\bf Mail} a message to someone.\\
    n       & Go to the {\bf next} group with unread articles. [distance]\\
    M-n     & Go to the {\bf next} group on the same or lower level.
    [distance]\\ 
    p       & (DEL) Go to the {\bf previous} group with unread articles.
    [distance]\\ 
    M-p     & Go to the {\bf previous} group on the same or lower level. [distance]\\ 
    q       & {\bf Quit} Gnus.\\
    r       & Re-read the init file ({\bf reset}).\\
    s       & {\bf Save} the `.newsrc.eld' file (and `.newsrc' if
    `gnus-save-newsrc-file').\\ 
    z       & Suspend (kill all buffers of) Gnus.\\
    B       & {\bf Browse} a foreign server.\\
    C       & Mark all articles in this group as read ({\bf Catch-up}). [p/p]\\
    F       & {\bf Find} new groups and process them.\\
    N       & Go to the {\bf next} group. [distance]\\
    P       & Go to the {\bf previous} group. [distance]\\
    Q       & {\bf Quit} Gnus without saving any startup (.newsrc) files.\\
    R       & {\bf Restart} Gnus.\\
    Z       & Clear the dribble buffer.\\
    M-c     & Clear data from group (marks and list of read articles). \\
    C-c C-s & {\bf Sort} the groups by name, number of unread articles, or level
    (depending on `gnus-group-sort-function').\\
    C-c C-x & Run all expirable articles in this group through the {\bf expiry} 
    process.\\
    C-c M-C-x & Run all articles in all groups through the {\bf expiry} process.\\
    C-c M-g & Activate all {\bf groups}.\\
    C-c C-i & Gnus online-manual ({\bf info}).\\
    C-x C-t & {\bf Transpose} two groups.\\
    H f     & Fetch this group's {\bf FAQ} (using ange-ftp).\\
    H v     & (V) Display the Gnus {\bf version} number.\\
    H d     & (C-c C-d) Show the {\bf description} of this group
    [Prefix: re-read from server].\\ 
    M-d     & {\bf Describe} all groups. [Prefix: re-read from server]\\
  \end{keys}
  }

\newcommand{\ListGroups}{%
  {\esamepage
    \begin{keys}{A M}
      A d     & List all groups whose names or {\bf descriptions} match a regexp.\\ 
      A k     & (C-c C-l) List all {\bf killed} groups.
      [Prefix: look at active-file from server]\\
      A l     & List all groups on a specific level.
      [Prefix: also list groups with no unread articles]\\
      A d     & List all groups that have names or {\bf descriptions} matching
      a regexp.\\
      A a     & (C-c C-a) List all groups whose names match a regexp
      ({\bf apropos}).\\ 
      A A     & List the server's active-file.\\
      A M     & List groups that {\bf match} a regexp.\\
      A m     & List groups that {\bf match} a regexp and have unread articles. 
      [level]\\ 
      A s     & (l) List all {\bf subscribed} groups with unread articles.
      [level; 5 and lower is the default]\\
      A u     & (L) List all groups (including read and {\bf unsubscribed}).
      [level; 7 and lower is the default]\\
      A z     & List all {\bf zombie} groups.\\
    \end{keys}
    }

  \newcommand{\CreateEditGroups}{%
    {\esamepage
      The select methods are indicated in parentheses.\\*
      \begin{keys}{G DEL}
        G a     & Make the Gnus list {\bf archive} group. (nndir over ange-ftp)\\
        G c     & {\bf Customize} this group's parameters.\\
        G d     & Make a {\bf directory} group (every file must be a posting and files
        must have numeric names). (nndir)\\
        G D     & Enter a {\bf directory} as a (temporary) group.
        (nneething without recording articles read)\\
        G e     & (M-e) {\bf Edit} this group's select method.\\
        G E     & {\bf Edit} this group's info (select method, articles read, etc).\\
        G f     & Make a group based on a {\bf file}. (nndoc)\\
        G h     & Make the Gnus {\bf help} (documentation) group. (nndoc)\\
        G k     & Make a {\bf kiboze} group. (nnkiboze)\\
        G m     & {\bf Make} a new group.\\
        G p     & Edit this group's {\bf parameters}.\\
        G r     & Rename this group (does not work with read-only groups!).\\
        G u     & Create one of the groups mentioned in gnus-{\bf useful}-groups.\\
        G v     & Add this group to a {\bf virtual} group. [p/p]\\
        G V     & Make a new empty {\bf virtual} group. (nnvirtual)\\
        G w     & Create ephemeral group based on web-search. [Prefix: make solid group
        instead]\\
        G DEL   & {\bf Delete} group [Prefix: delete all articles as well].\\
      \end{keys}
      You can also create mail-groups and read your mail with Gnus (very useful
      if you are subscribed to mailing lists), using one of the methods
      nnmbox, nnbabyl, nnml, nnmh, or nnfolder. Read about it in the online info
      (C-c C-i g Reading Mail RET).
      }}

                                % TODO:
  \newcommand{\SoupCommands}{%
    \begin{keys}{G s w}
      G s b   & gnus-group-brew-soup: not documented.\\
      G s p   & gnus-soup-pack-packet: not documented.\\
      G s r   & nnsoup-pack-replies: not documented.\\
      G s s   & gnus-soup-send-replies: not documented.\\
      G s w   & gnus-soup-save-areas: not documented.\\
    \end{keys}}

  \newcommand{\MarkGroups}{%
    \begin{keys}{M m}
      M m     & (\#) Set the process {\bf mark} on this group. [scope]\\
      M r     & Mark all groups matching regular expression.\\
      M u     & (M-\#) Remove the process mark from this group ({\bf unmark}).
      [scope]\\ 
      M U     & Remove the process mark from all groups (\textbf{umark all}).\\
      M w     & Mark all groups in the current region.\\
    \end{keys}}

  \newcommand{\GroupTopicsGeneral}{%
    {\esamepage
      Topics are ``categories'' for groups. Press t in the group-buffer to
      toggle gnus-topic-mode (C-c C-i g Group Topics RET).
      \begin{keys}{C-c C-x}
        T n     & Prompt for topic {\bf name} and create it.\\
        T m     & {\bf Move} the current group to some other topic [p/p].\\
        T j     & {\bf Jump} to a topic.\\
        T c     & {\bf Copy} the current group to some other topic [p/p].\\
        T D     & Remove (not delete) the current group [p/p].\\
        T M     & {\bf Move} all groups matching a regexp to a topic.\\
        T C     & {\bf Copy} all groups matching a regexp to a topic.\\
        T H     & Toggle {\bf hiding} of empty topics.\\
        T r     & {\bf Rename} a topic.\\
        T DEL   & Delete an empty topic.\\
        T \#    & Mark all groups in the current topic with the process-mark.\\
        T M-\#  & Remove the process-mark from all groups in the current topic.\\
        T TAB   & (TAB) Indent current topic [Prefix: unindent].\\ 
        M-TAB   & Unindent the current topic.\\
        RET     & (SPC) Either unfold topic or enter group [level].\\
        C-c C-x & Expire all articles in current group or topic.\\
        C-k     & {\bf Kill} a group or topic.\\
        C-y     & {\bf Yank} a group or topic.\\
        A T     & List active-file using {\bf topics}.\\
        G p     & Edit topic-{\bf parameters}.\\
      \end{keys}
      }
    }

  \newcommand{\TopicSorting}{%
    {\esamepage
      \begin{keys}{T S m}
        T S a  & Sort {\bf alphabetically}.\\
        T S u  & Sort by number of {\bf unread} articles.\\
        T S l  & Sort by group {\bf level}.\\
        T S v  & Sort by group score ({\bf value}).\\
        T S r  & Sort by group {\bf rank}.\\
        T S m  & Sort by {\bf method}.\\
      \end{keys}
      }
    }

  \newcommand{\SubscribeKillYankGroups}{%
    {\esamepage
      \begin{keys}{S C-k}
        S k     & (C-k) {\bf Kill} this group.\\
        S l     & Set the {\bf level} of this group. [p/p]\\
        S s     & (U) Prompt for a group and toggle its {\bf subscription}.\\
        S t     & (u) {\bf Toggle} subscription to this group. [p/p]\\
        S w     & (C-w) Kill all groups in the region.\\
        S y     & (C-y) {\bf Yank} the last killed group.\\
        S z     & Kill all {\bf zombie} groups.\\
        S C-k   & Kill all groups on a certain level.\\
      \end{keys}
      }
    }

  \newcommand{\SummaryModeGeneral}{%
    {\esamepage
      \begin{keys}{M-RET}
        SPC     & (A SPC, A n) Select an article, scroll it one page, move to the
        next one.\\ 
        DEL     & (A DEL, A p, b) Scroll this article one page back. [distance]\\
        RET     & Scroll this article one line forward. [distance]\\
        M-RET   & Scroll this article one line backward. [distance]\\
        =       & Expand the Summary window (fullsize).
        [Prefix: shrink to display article window]\\
                                %
        \&      & Execute a command on all articles whose header matches a regexp.
        [Prefix: move backwards]\\
        M-\&    & Execute a command on all articles having the process mark.\\
                                %
        M-n     & (G M-n) Go to the {\bf next} summary line of an unread article.
        [distance]\\ 
        M-p     & (G M-p) Go to the {\bf previous} summary line of an unread article. 
        [distance]\\ 
        M-s     & {\bf Search} through all subsequent articles for a regexp.\\
        M-r     & Search through all previous articles for a regexp.\\
                                %
        A P     & {\bf Postscript}-print current buffer.\\
                                %
        M-k     & Edit this group's {\bf kill} file.\\
        M-K     & Edit the general {\bf kill} file.\\
                                %
        C-t     & Toggle {\bf truncation} of summary lines.\\
        Y g     & Regenerate the summary-buffer.\\
        Y c     & Insert all cached articles into the summary-buffer.\\
                                %
        M-C-e   & {\bf Edit} the group-parameters.\\
        M-C-g   & Customize the group-parameters.\\
                                %
                                % article handling
                                %
        A $<$   & ($<$, A b) Scroll to the beginning of this article.\\
        A $>$   & ($>$, A e) Scroll to the end of this article.\\
        A s     & (s) Perform an i{\bf search} in the article buffer.\\
                                %
        A D     & (C-d) Un{\bf digestify} this article into a separate group.
        [Prefix: force digest]\\
        M-C-d   & Like C-d, but open several documents in nndoc-groups, wrapped
        in an nnvirtual group [p/p]\\
                                %
        A g     & (g) (Re)fetch this article ({\bf get}). [Prefix: get raw version]\\ 
        A r     & (\^{}, A \^{}) Fetch the parent(s) of this article.
        [Prefix: if positive fetch \textit{n} ancestors;
        negative: fetch only the \textit{n}th ancestor]\\
        A t     & {\bf Translate} this article.\\
        A R     & Fetch all articles mentioned in the {\bf References}-header.\\
        A T     & Fetch full \textbf{thread} in which the current article appears.\\
        M-\^{}   & Fetch the article with a given Message-ID.\\
        S y     & {\bf Yank} the current article into an existing message-buffer.
        [p/p]\\
      \end{keys}
      }
    }

  \newcommand{\MIMESummary}{%
    {\esamepage
      For the commands operating on one MIME part (a subset of gnus-article-*), a
      prefix selects which part to operate on. If the point is placed over a
      MIME button in the article buffer, use the corresponding bindings for the
      article buffer instead.
      \begin{keys}{W M w}
        K v      & (b) {\bf View} the MIME-part.\\
        K o      & {\bf Save} the MIME part.\\
        K c      & {\bf Copy} the MIME part.\\
        K e      & View the MIME part {\bf externally}.\\
        K i      & View the MIME part {\bf internally}.\\
        K $\mid$ & Pipe the MIME part to an external command.\\
        K b      & Make all the MIME parts have buttons in front of them.\\
        K m      & Try to repair {\bf multipart-headers}.\\
        X m      & Save all parts matching a MIME type to a directory. [p/p]\\
        M-t      & Toggle the buttonized display of the article buffer.\\
        W M w    & Decode RFC2047-encoded words in the article headers.\\
        W M c    & Decode encoded article bodies. [Prefix: prompt for charset]\\
        W M v    & View all MIME parts in the current article.\\
      \end{keys}
      }
    }

  \newcommand{\SortSummary}{%
    {\esamepage
      \begin{keys}{C-c C-s C-a}
        C-c C-s C-a & Sort the summary-buffer by {\bf author}.\\
        C-c C-s C-d & Sort the summary-buffer by {\bf date}.\\
        C-c C-s C-i & Sort the summary-buffer by article score.\\
        C-c C-s C-l & Sort the summary-buffer by amount of lines.\\
        C-c C-s C-c & Sort the summary-buffer by length.\\
        C-c C-s C-n & Sort the summary-buffer by article {\bf number}.\\
        C-c C-s C-s & Sort the summary-buffer by {\bf subject}.\\
      \end{keys}
      }
    }

  \newcommand{\MailGroups}{% formerly \Bsubmap
    {\esamepage
      These commands (except `B c') are only valid in a mail group.\\*
      \begin{keys}{B M-C-e}
        B DEL   & (B backspace, B delete) {\bf Delete} the mail article from disk (!).
        [p/p]\\
        B B     & Crosspost this article to another group.\\
        B c     & {\bf Copy} this article from any group to a mail group. [p/p]\\
        B e     & {\bf Expire} all expirable articles in this group. [p/p]\\
        B i     & {\bf Import} a random file into this group.\\
        B m     & {\bf Move} the article from one mail group to another. [p/p]\\
        B p     & Query whether the article was posted as well.\\
        B q     & {\bf Query} where the article will end up after fancy splitting\\
        B r     & {\bf Respool} this mail article. [p/p]\\
        B t     & {\bf Trace} the fancy splitting patterns applied to this article.\\
        B w     & (e) Edit this article.\\
        B M-C-e & {\bf Expunge} (delete from disk) all expirable articles in this group
        (!). [p/p]\\ 
      \end{keys}
      }
    }

  \newcommand{\DraftGroup}{% formerly \Dsubmap
    {\esamepage
      The ``drafts''-group contains messages that have been saved but not sent
      and rejected articles. \\*
      \begin{keys}{B DEL}
        D e      & \textbf{edit} message.\\
        D s      & \textbf{Send} message. [p/p]\\
        D S      & \textbf{Send} all messages.\\
        D t      & \textbf{Toggle} sending (mark as unsendable).\\
        B DEL    & \textbf{Delete} message (like in mailgroup).\\
      \end{keys}
      }
    }

  \newcommand{\SelectArticles}{% formerly \Gsubmap
    {\esamepage
      These commands select the target article. They do not understand the prefix.\\*
      \begin{keys}{G C-n}
        h       & Enter article-buffer.\\
        G b     & (,) Go to the {\bf best} article (the one with highest score).\\
        G f     & (.) Go to the {\bf first} unread article.\\
        G n     & (n) Go to the {\bf next} unread article.\\
        G p     & (p) Go to the {\bf previous} unread article.\\
                                %
        G N     & (N) Go to {\bf the} next article.\\
        G P     & (P) Go to the {\bf previous} article.\\
                                %
        G C-n   & (M-C-n) Go to the {\bf next} article with the same subject.\\
        G C-p   & (M-C-p) Go to the {\bf previous} article with the same subject.\\
                                %
        G l     & (l) Go to the previously read article ({\bf last-read-article}).\\
        G o     & Pop an article off the summary history and go to it.\\
                                %
        G g     & Search an article via subject.\\
        G j     & (j) Search an article via Message-Id or subject.\\
      \end{keys}
      }
    }

  \newcommand{\ArticleModeGeneral}{%
    {\esamepage
      The normal navigation keys work in Article mode. Some additional keys are:\\
      \begin{keys}{C-c RET}
        C-c \^{} & Get the article with the Message-ID near point.\\
        C-c RET & Send reply to address near point.\\
        h       & Go to the \textbf{header}-line of the article in the
        summary-buffer.\\
        s       & Go to \textbf{summary}-buffer.\\
        RET     & (middle mouse button) Activate the button at point to follow
        an URL or Message-ID.\\
        TAB     & Move the point to the next button.\\
        M-TAB   & Move point to previous button.\\
      \end{keys}
      }
    }

  \newcommand{\WashArticle}{% formerly \Wsubmap
    {\esamepage
      \begin{keys}{W W H}
        W b     & Make Message-IDs and URLs in the article mouse-clickable
        {\bf buttons}.\\  
        W l     & (w) Remove page breaks ({\bf\^{}L}) from the article.\\
        W c     & Translate CRLF-pairs to LF and then the remaining CR's to LF's.\\
        W d     & Treat {\bf dumbquotes}.\\
        W f     & Look for and display any X-{\bf Face} headers.\\
        W m     & Toggle {\bf MIME} processing.\\
        W o     & Treat {\bf overstrike} or underline (\^{}H\_) in the article.\\
        W q     & Treat {\bf quoted}-printable in the article.\\
        W r     & (C-c C-r) Do a Caesar {\bf rotate} (rot13) on the article.\\
        W t     & (t) {\bf Toggle} display of all headers.\\
        W v     & (v) Toggle permanent {\bf verbose} displaying of all headers.\\
        W w     & Do word {\bf wrap} in the article.\\
        W B     & Add clickable {\bf buttons} to the article headers.\\
        W C     & {\bf Capitalize} first word in each sentence.\\
        W Q     & Fill long lines.\\
                                %
        W W H   & Strip certain {\bf headers} from body.\\
                                %
        W E l   & Strip blank {\bf lines} from the beginning of the article.\\
        W E m   & Replace blank lines with empty lines and remove {\bf multiple}
        blank lines.\\
        W E t   & Remove {\bf trailing} blank lines.\\
        W E a   & Strip blank lines at the beginning and the end
        (W E l, W E m and W E t).\\
        W E A   & Strip {\bf all} blank lines.\\
        W E s   & Strip leading blank lines from the article body.\\
        W E e   & Strip trailing blank lines from the article body.\\
                                %
        W T u   & (W T z) Display the article timestamp in GMT ({\bf UT, ZULU}).\\
        W T i   & Display the article timestamp in {\bf ISO} 8601.\\
        W T l   & Display the article timestamp in the {\bf local} timezone.\\
        W T s   & Display according to `gnus-article-time-format'.\\
        W T e   & Display the time {\bf elapsed} since it was sent.\\
        W T o   & Display the {\bf original} timestamp.\\
      \end{keys}
      }
    }

  \newcommand{\HideHighlightArticle}{%
    {\esamepage
      \begin{keys}{W W C-c}
        W W a   & Hide {\bf all} unwanted parts. Calls W W h, W W s, W W C-c.\\
        W W h   & Hide article {\bf headers}.\\
        W W b   & Hide {\bf boring} headers.\\
        W W s   & Hide {\bf signature}.\\
        W W l   & Hide {\bf list} identifiers in subject-header.\\
        W W p   & Hide {\bf PGP}-signatures.\\
        W W P   & Hide {\bf PEM} (privacy enhanced messages).\\
        W W B   & Hide banner specified by group parameter.\\
        W W c   & Hide {\bf citation}.\\
        W W C-c & Hide {\bf citation} using a more intelligent algorithm.\\
        W W C   & Hide cited text in articles that aren't roots.\\
                                %
        W e     & {\bf Emphasize} article.\\
                                %
        W H a   & Highlight {\bf all} parts. Calls W b, W H c, W H h, W H s.\\
        W H c   & Highlight article {\bf citations}.\\
        W H h   & Highlight article {\bf headers}.\\
        W H s   & Highlight article {\bf signature}.\\
      \end{keys}
      For all hiding-commands: A positive prefix always hides, and a negative
      prefix will show what was previously hidden.
      }}

  \newcommand{\MIMEArticleMode}{%
    {\esamepage
      \begin{keys}{M-RET}
        RET     & (BUTTON-2) Toggle display of the MIME object.\\
        v       & (M-RET) Prompt for a method and then view object using this method.\\
        o       & Prompt for a filename and save the MIME object.\\
        c       & {\bf Copy} the MIME object to a new buffer and display this buffer.\\
        t       & View the MIME object as a different {\bf type}.\\
        $\mid$  & Pipe the MIME object to a process.\\
      \end{keys}
      }
    }

  %% end of article mode for reading ..........................................

  \newcommand{\MarkArticlesGeneral}{% formerly \Msubmap
    {\esamepage
      \begin{keys}{M M-C-r}
        d       & (M d, M r) Mark this article as read and move to the next one.
        [scope]\\ 
        D       & Mark this article as read and move to the previous one. [scope]\\
        !       & (u, M !, M t) Tick this article (mark it as interesting) and move
        to the next one. [scope]\\
        U       & Tick this article and move to the previous one. [scope]\\ 
        M ?     & (?) Mark this article as dormant (only followups are
        interesting). [scope]\\ 
        M D     & Show all {\bf dormant} articles (normally they are hidden unless they
        have any followups).\\
        M M-D   & Hide all {\bf dormant} articles.\\
        C-w     & Mark all articles between point and mark as read.\\
        M-u     & (M SPC, M c) Clear all marks from this article and move to the next
        one. [scope]\\ 
        M-U     & Clear all marks from this article and move to the previous one.
        [scope]\\
                                %
        M e     & (E, M x) Mark this article as {\bf expirable}. [scope]\\
                                %
        M k     & (k) {\bf Kill} all articles with the same subject then select the
        next unread one.\\ 
        M K     & (C-k) {\bf Kill} all articles with the same subject as this one.\\
                                %
        M C     & {\bf Catch-up} the articles that are not ticked and not dormant.\\
        M C-c   & {\bf Catch-up} all articles in this group.\\
        M H     & {\bf Catch-up} (mark read) this group to point (to-{\bf here}).\\
                                %
        M b     & Set a {\bf bookmark} in this article.\\
        M B     & Remove the {\bf bookmark} from this article.\\
                                %
        M M-r   & (x) Expunge all {\bf read} articles from this group.\\
        M M-C-r & Expunge all articles having a given mark.\\
        M S     & (C-c M-C-s) {\bf Show} all expunged articles.\\
        M M C-h & Displays some more keys doing ticking slightly differently.\\
      \end{keys}
      The variable `gnus-summary-goto-unread' controls what happens after a mark
      has been set (C-x C-i g Setting Marks RET)
      }}

  \newcommand{\MarkByScore}{%
    \begin{keys}{M V m}
      M V c   & {\bf Clear} all marks from all high-scored articles. [score]\\
      M V k   & {\bf Kill} all low-scored articles. [score]\\
      M V m   & Mark all high-scored articles with a given {\bf mark}. [score]\\
      M V u   & Mark all high-scored articles as interesting (tick them). [score]\\
    \end{keys}
    }
  }

\newcommand{\ProcessMark}{%
  {\esamepage 
    These commands set and remove the process mark (\#). You only need to use
    it if the set of articles you want to operate on is non-contiguous. Else
    use a numeric prefix.\\*
    \begin{keys}{M P R}
      M P p   & (\#, M \#) Mark this article.\\
      M P u   & (M-\#, M M-\#) \textbf{unmark} this article.\\
      M P b   & Mark all articles in {\bf buffer}.\\
      M P r   & Mark all articles in the {\bf region}.\\
      M P R   & Mark all articles matching a {\bf regexp}.\\
      M P t   & Mark all articles in this (sub){\bf thread}.\\
      M P s   & Mark all articles in the current {\bf series}.\\
      M P S   & Mark all {\bf series} that already contain a marked article.\\
      M P a   & Mark {\bf all} articles (in series order).\\
      M P U   & \textbf{unmark} all articles.\\
                                %
      M P i   & {\bf Invert} the list of process-marked articles.\\
      M P k   & Push the current process-mark set onto stack and unmark
      all articles.\\
      M P y   & Pop process-mark set from stack and restore it.\\
    \end{keys}
    }
  }

\newcommand{\Limiting}{%
  {\esamepage
    \begin{keys}{/M}
      //   & (/s) Limit the summary-buffer to articles matching {\bf subject}.\\
      /a   & Limit the summary-buffer to articles matching {\bf author}.\\
      /x   & Limit depending on ``extra'' headers.\\
      /u   & (x) Limit to {\bf unread} articles.
      [Prefix: also exclude ticked and dormant articles]\\
      /m   & Limit to articles marked with specified {\bf mark}.\\
      /t   & Ask for a number and exclude articles younger than that many days.
      [Prefix: exclude older articles]\\
      /n   & Limit to current article. [p/p]\\
      /w   & Pop the previous limit off the stack and restore it.
      [Prefix: pop all limits]\\
      /v   & Limit to score. [score]\\
      /E   & (M S) Include all expunged articles in the limit.\\
      /D   & Include all dormant articles in the limit.\\
      /*   & Limit to cached articles.\\
      Y C  & Include all cached articles in the limit.\\
      /d   & Exclude all dormant articles from the limit.\\
      /M   & Exclude all marked articles.\\
      /T   & Include all articles from the current thread in the limit.\\
      /c   & Exclude all dormant articles that have no children from the limit.\\
      /C   & Mark all excluded unread articles as read.
      [Prefix: also mark ticked and dormant articles]\\
    \end{keys}
    }
  }

\newcommand{\OutputArticles}{% formerly \Osubmap
  {\esamepage
    \begin{keys}{O m}
      O o     & (o, C-o) Save this article using the default article saver. [p/p]\\
      O b     & Save this article's {\bf body} in plain file format [p/p]\\
      O f     & Save this article in plain {\bf file} format. [p/p]\\
      O F     & like O f, but overwrite file's contents. [p/p]\\
      O h     & Save this article in {\bf mh} folder format. [p/p]\\
      O m     & Save this article in {\bf mail} format. [p/p]\\
      O r     & Save this article in {\bf rmail} format. [p/p]\\
      O v     & Save this article in {\bf vm} format. [p/p]\\
      O p     & ($\mid$) Pipe this article to a shell command. [p/p]\\
    \end{keys}
    }
  }

\newcommand{\PostReplyetc}{% formerly \Ssubmap
  {\esamepage
    These commands put you in a separate news or mail buffer. See the section
    about composing messages for more information.
                                %After
                                %editing the article, send it by pressing C-c C-c.  If you are in a
                                %foreign group and want to post the article using the foreign server, give
                                %a prefix to C-c C-c.\\* 
    \begin{keys}{S O m}
      S p     & (a) {\bf Post} an article to this group.\\
      S f     & (f) Post a {\bf followup} to this article.\\
      S F     & (F) Post a {\bf followup} and include the original. [p/p]\\
      S o p   & Forward this article as a {\bf post} to a newsgroup.
      [Prefix: include all headers]\\
      S M-c   & Send a complaint about excessive crossposting to the author of this
      article. [p/p]\\
                                %
      S m     & (m) Send {\bf a} mail to some other person.\\
      S r     & (r) Mail a {\bf reply} to the author of this article.\\
      S R     & (R) Mail a {\bf reply} and include the original. [p/p]\\
      S w     & Mail a {\bf wide} reply to this article.\\
      S W     & Mail a {\bf wide} reply to this article.\\
      S o m   & (C-c C-f) Forward this article by {\bf mail} to a person.
      [Prefix: include all headers]\\
      S D b   & Resend {\bf bounced} mail.\\
      S D r   & {\bf Resend} mail to a different person.\\
                                %
      S n     & Post a followup via {\bf news} even if you got the message
      through mail.\\
      S N     & Post a followup via {\bf news} and include the original mail.
      [p/p]\\
                                %
      S c     & (C) {\bf Cancel} this article (only works if it is your own).\\
      S s     & {\bf Supersede} this article with a new one (only for own
      articles).\\
                                %
      S O m   & Digest these series and forward by {\bf mail}. [p/p]\\
      S O p   & Digest these series and forward as a {\bf post} to a newsgroup.
      [p/p]\\ 
                                %
      S u     & {\bf Uuencode} a file and post it as a series.\\
    \end{keys}
    If you want to cancel or supersede an article you just posted (before it
    has appeared on the server), go to the *post-news* buffer, change
    `Message-ID' to `Cancel' or `Supersedes' and send it again with C-c C-c.
    }}

\newcommand{\Threading}{% formerly \Tsubmap
  {\esamepage
    \begin{keys}{T M-\#}
      T \#    & Mark this thread with the process mark.\\
      T M-\#  & Remove process-marks from this thread.\\
                                %
      T t     & Re-{\bf thread} the current article's thread.\\
                                % movement
      T n     & (M-C-f) Go to the {\bf next} thread. [distance]\\
      T p     & (M-C-b) Go to the {\bf previous} thread. [distance]\\
      T d     & {\bf Descend} this thread. [distance]\\
      T u     & Ascend this thread ({\bf up}-thread). [distance]\\
      T o     & Go to the top of this thread.\\
                                %
      T s     & {\bf Show} the thread hidden under this article.\\
      T h     & {\bf Hide} this (sub)thread.\\
                                %
      T i     & {\bf Increase} the score of this thread.\\
      T l     & (M-C-l) {\bf Lower} the score of this thread.\\
                                %
      T k     & (M-C-k) {\bf Kill} the current (sub)thread. [Negative prefix:
      tick it, positive prefix: unmark it.]\\
                                %
      T H     & {\bf Hide} all threads.\\
      T S     & {\bf Show} all hidden threads.\\
      T T     & (M-C-t) {\bf Toggle} threading.\\
    \end{keys}
    }
  }

\newcommand{\Scoring}{% formerly \Vsubmap
  {\esamepage
    Read about Adaptive Scoring in the online info.\\*
    \begin{keys}{\bf A p m l}
      V a     & {\bf Add} a new score entry, specifying all elements.\\
      V c     & Specify a new score file as {\bf current}.\\
      V e     & {\bf Edit} the current score alist.\\
      V f     & Edit a score {\bf file} and make it the current one.\\
      V m     & {\bf Mark} all articles below a given score as read.\\
      V s     & Set the {\bf score} of this article.\\
      V t     & Display all score rules applied to this article ({\bf track}).\\
      V x     & {\bf Expunge} all low-scored articles. [score]\\
      V C     & {\bf Customize} the current score file through a user-friendly
      interface.\\ 
      V S     & Display the {\bf score} of this article.\\
      \bf A p m l& Make a scoring entry based on this article.\\
    \end{keys}
    The four letters stand for:\\*
    \quad \B{A}ction: I)ncrease, L)ower;\\*
    \quad \B{p}art: a)utor (from), s)ubject, x)refs (cross-posting), d)ate, l)ines,
    message-i)d, t)references (parent), f)ollowup, b)ody, h)ead (all headers);\\*
    \quad \B{m}atch type:\\*
    \qquad string: s)ubstring, e)xact, r)egexp, f)uzzy,\\*
    \qquad date: b)efore, a)t, n)this,\\*
    \qquad number: $<$, =, $>$;\\*
    \quad \B{l}ifetime: t)emporary, p)ermanent, i)mmediate.

    If you type the second letter in uppercase, the remaining two are assumed
    to be s)ubstring and t)emporary. 
    If you type the third letter in uppercase, the last one is assumed to be 
    t)emporary.

    \quad Extra keys for manual editing of a score file:\\*
    \begin{keys}{C-c C-c}
      C-c C-c & Finish editing the score file.\\
      C-c C-d & Insert the current {\bf date} as number of days.\\
    \end{keys}
    }
  }

\newcommand{\ExtractSeries}{% formerly \Xsubmap
  {\esamepage
    Gnus recognizes if the current article is part of a series (multipart
    posting whose parts are identified by numbers in their subjects, e.g.{}
    1/10\dots10/10) and processes the series accordingly. You can mark and
    process more than one series at a time. If the posting contains any
    archives, they are expanded and gathered in a new group.\\*
    \begin{keys}{X p}
      X b     & Un-{\bf binhex} these series. [p/p]\\
      X o     & Simply {\bf output} these series (no decoding). [p/p]\\ 
      X p     & Unpack these {\bf postscript} series. [p/p]\\
      X s     & Un-{\bf shar} these series. [p/p]\\
      X u     & {\bf Uudecode} these series. [p/p]\\
    \end{keys}

    Each one of these commands has four variants:\\*
    \begin{keys}{X v \bf Z}
      X   \bf z & Decode these series. [p/p]\\
      X   \bf Z & Decode and save these series. [p/p]\\
      X v \bf z & Decode and view these series. [p/p]\\
      X v \bf Z & Decode, save and view these series. [p/p]\\
    \end{keys}
    where {\bf z} or {\bf Z} identifies the decoding method (b, o, p, s, u).

    An alternative binding for the most-often used of these commands is\\*
    \begin{keys}{C-c C-v C-v}
      C-c C-v C-v & (X v u) Uudecode and view these series. [p/p]\\
    \end{keys}
    }}

\newcommand{\ExitSummary}{% formerly \Zsubmap
  {\esamepage
    \begin{keys}{Z G}
      Z Z     & (q, Z Q) Exit this group.\\
      Z E     & (Q) {\bf Exit} without updating the group information.\\
                                %
      Z c     & (c) Mark all unticked articles as read ({\bf catch-up}) and exit.\\
      Z C     & Mark all articles as read ({\bf catch-up}) and exit.\\
                                %
      Z n     & Mark all articles as read and go to the {\bf next} group.\\
      Z N     & Exit and go to {\bf the} next group.\\
      Z P     & Exit and go to the {\bf previous} group.\\
                                %
      Z G     & (M-g) Check for new articles in this group ({\bf get}).\\
      Z R     & Exit this group, and then enter it again ({\bf reenter}).
      [Prefix: select all articles, read and unread.]\\
      Z s     & Update and save the dribble buffer. [Prefix: save .newsrc* as well]\\
    \end{keys}
    }
  }

\newcommand{\MsgCompositionGeneral}{%
  Press C-c ? in the composition-buffer to get this information.
  {\esamepage
    \begin{keys}{C-c C-m}
                                % sending
      C-c C-c & Send message and exit. [Prefix: send via foreign server]\\
      C-c C-s & Send message. [Prefix: send via foreign server]\\
      C-c C-d & Don't send message (save as \textbf{draft}).\\
      C-c C-k & \textbf{Kill} message-buffer.\\
      C-c C-m & {\bf Mail} reply to the address near point.
      [Prefix: include the original]\\
                                % modify headers/body
      C-c C-t & Paste the recipient's address into \textbf{To:}-field.\\
      C-c C-n & Insert a \textbf{Newsgroups:}-header.\\
      C-c C-o & Sort headers.\\
      C-c C-e & \textbf{Elide} region.\\
      C-c C-v & Kill everything outside region.\\
      C-c C-r & Do a \textbf{Rot-13} on the body.\\
      C-c C-w & Insert signature (from `message-signature-file').\\
      C-c C-z & Kill everything up to signature.\\
      C-c C-y & \textbf{Yank} original message.\\
      C-c C-q & Fill the yanked message.\\
    \end{keys}
    }
  }

\newcommand{\MsgCompositionMovementArticle}{%
  The following functions create the header-field if necessary.
  {\esamepage
    \begin{keys}{C-c C-f C-u}
      C-c TAB & Move to \textbf{signature}.\\
      C-c C-b & Move to \textbf{body}.\\
      C-c C-f C-t & Move to \textbf{To:}.\\
      C-c C-f C-c & Move to \textbf{Cc:}.\\
      C-c C-f C-b & Move to \textbf{Bcc:}.\\
      C-c C-f C-w & Move to \textbf{Fcc:}.\\
      C-c C-f C-s & Move to \textbf{Subject:}.\\
      C-c C-f C-r & Move to \textbf{Reply-To:}.\\
      C-c C-f C-f & Move to \textbf{Followup-To:}.\\
      C-c C-f C-n & Move to \textbf{Newsgroups:}.\\
      C-c C-f C-u & Move to \textbf{Summary:}.\\
      C-c C-f C-k & Move to \textbf{Keywords:}.\\
      C-c C-f C-d & Move to \textbf{Distribution:}.\\
    \end{keys}
    }
  }

\newcommand{\MsgCompositionMML}{%
  {\esamepage
    \begin{keys}{C-c C-m P}
      C-c C-m f   & (C-c C-a) Attach \textbf{file}.\\
      C-c C-m b   & Attach contents of \textbf{buffer}.\\
      C-c C-m e   & Attach \textbf{external} file (ftp..).\\
      C-c C-m P   & Create MIME-\textbf{preview} (new buffer).\\
      C-c C-m v   & \textbf{Validate} article.\\
      C-c C-m p   & Insert \textbf{part}.\\
      C-c C-m m   & Insert \textbf{multi}-part.\\
      C-c C-m q   & \textbf{Quote} region.\\
                                % TODO: narrow headers (C-c C-m n) ?
    \end{keys}
    }
  }

%% TODO:
\newcommand{\ServerMode}{%
  {\esamepage
    To enter this mode, press `\^' while in Group mode.\\*
    \begin{keys}{SPC}
      SPC     & (RET) Browse this server.\\
      a       & {\bf Add} a new server.\\
      c       & {\bf Copy} this server.\\
      e       & {\bf Edit} a server.\\
      k       & {\bf Kill} this server. [scope]\\
      l       & {\bf List} all servers.\\
      q       & Return to the group buffer ({\bf quit}).\\
      s       & Request that the server scan its sources for new articles.\\
      g       & Request that the server regenerate its data.\\
      y       & {\bf Yank} the previously killed server.\\
    \end{keys}
    }
  }

\newcommand{\BrowseServer}{%
  {\esamepage
    To enter this mode, press `B' while in Group mode.\\*
    \begin{keys}{RET}
      RET     & Enter the current group.\\
      SPC     & Enter the current group and display the first article.\\
      ?       & Give a very short help message.\\
      n       & Go to the {\bf next} group. [distance]\\
      p       & Go to the {\bf previous} group. [distance]\\
      q       & (l) {\bf Quit} browse mode.\\
      u       & Subscribe to the current group. [scope]\\
    \end{keys}
    }
  }

\newcommand{\GroupUnplugged}{%
  {\esamepage
    \begin{keys}{J S}
      J j & Toggle plugged-state.\\
      J s & Fetch articles from current group.\\
      J s & Fetch articles from all groups for offline-reading.\\
      J S & \textbf{Send} all sendable messages in the drafts group.\\
                                %
      J c & Enter \textbf{category} buffer.\\
      J a & \textbf{Add} this group to an Agent category [p/p].\\
      J r & \textbf{Remove} this group from its Agent category [p/p].\\
    \end{keys}
    }
  }

\newcommand{\SummaryUnplugged}{%
  {\esamepage
    \begin{keys}{J M-\#}
      J \#   & \textbf{Mark} the article for downloading.\\
      J M-\# & \textbf{Unmark} the article for downloading.\\
      @      & \textbf{Toggle} whether to download the article.\\
      J c    & Mark all undownloaded articles as read (\textbf{catch-up}).\\
    \end{keys}
    }
  }

\newcommand{\ServerUnplugged}{%
  {\esamepage
    \begin{keys}{J a}
      J a & \textbf{Add} the current server to the list of servers covered
      by the agent.\\
      J r & \textbf{Remove} the current server from the list of servers covered
      by the agent.\\
    \end{keys}
    }
  }


\Title
\par
\Logo{refcard}
\Notes
%
%
\section*{Group-Mode}
\GroupModeGeneral
    \subsection*{Group Subscribedness-Levels}
    \GroupLevels
    \subsection*{List Groups}
    \ListGroups
    \subsection*{Create/Edit Foreign Groups}
    \CreateEditGroups
    \subsection*{Unsubscribe, Kill and Yank Groups}
    \SubscribeKillYankGroups
    \subsection*{Mark Groups}
    \MarkGroups
    \subsection*{Group-Unplugged}
    \GroupUnplugged
% topics in group-mode
    \subsection*{Group Topics}
    \GroupTopicsGeneral
    \subsubsection*{Topic Sorting}
    \TopicSorting
%
% summary-mode
\section*{Summary-Mode}
\SummaryModeGeneral
    \subsection*{Select Articles}
    \SelectArticles
%
    \subsection*{Threading}
    \Threading
%
    \subsection*{Limiting}
    \Limiting
    \subsection*{Sort the Summary-Buffer}
    \SortSummary
    \subsection*{Score (Value) Commands}
    \Scoring
% 
    \subsection*{MIME operations from the Summary-Buffer}
    \MIMESummary
    \subsection*{Extract Series (Uudecode etc)}
    \ExtractSeries
    \subsection*{Output Articles}
    \OutputArticles
%
    \subsection*{Post, Followup, Reply, Forward, Cancel}
    \PostReplyetc
    \subsection*{Message-Composition}
    \MsgCompositionGeneral
        \subsubsection*{Jumping in message-buffer}
        \MsgCompositionMovementArticle
        \subsubsection*{Attachments/MML}
        \MsgCompositionMML
% marking articles
    \subsection*{Mark Articles}
    \MarkArticlesGeneral
        \subsubsection*{Mark Based on Score}
        \MarkByScore
        \subsubsection*{The Process Mark}
        \ProcessMark
        \subsubsection*{Mark Indication-Characters}
        \MarkCharacters
%
    \subsection*{Summary-Unplugged}
    \SummaryUnplugged
    \subsection*{Mail-Group Commands}
    \MailGroups
    \subsection*{Draft-Group Commands}
    \DraftGroup
% exiting
    \subsection*{Exit the Summary-Buffer}
    \ExitSummary
%
%
\section*{Article Mode (reading)}
\ArticleModeGeneral
    \subsection*{Wash the Article-Buffer}
    \WashArticle
    \subsection*{Hide/Highlight Parts of the Article}
    \HideHighlightArticle
    \subsection*{MIME operations from the Article-Buffer (reading)}
    \MIMEArticleMode
%
%
\section*{Server Mode}
\ServerMode
    \subsection*{Unplugged-Server}
    \ServerUnplugged
%
%
\section*{Browse Server Mode}
\BrowseServer

%\pagebreak
\vspace*{\fill}
\Copyright

\end{document}


